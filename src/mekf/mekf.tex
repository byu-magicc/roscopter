\documentclass[english]{article}
\usepackage[T1]{fontenc}
\usepackage[latin9]{inputenc}
\usepackage{geometry}
\geometry{verbose,tmargin=1in,bmargin=1in,lmargin=1in,rmargin=1in}
\setlength{\parskip}{\medskipamount}
\setlength{\parindent}{0pt}
\usepackage{amsmath}

\makeatletter

\setcounter{MaxMatrixCols}{20}

\makeatother

\usepackage{babel}
\begin{document}

\title{Indirect Multiplicative Extended Kalman Filter}

\author{Jerel Nielsen}
\maketitle



\section{Nomenclature}

\begin{tabbing}
	XXXXX \= \kill% this line sets tab stop$p_x$ \> Forward direction \\
	$\bf{x}$ \> State \\
	$\delta {\bf x}$ \> Error state \\
	$P$ \> Covariance of the error state \\
	${\bf{p}}_b$ \> Position of the quadrotor w.r.t. the inertial frame expressed in the inertial frame \\
	${\bf{q}}_b$ \> Attitude of the quadrotor w.r.t. the inertial frame expressed in the body frame \\
	${\bf{v}}_b$ \> Velocity of the quadrotor w.r.t. the inertial frame expressed in the body frame \\
	$\boldsymbol{\omega}$ \> Angular velocity of the quadrotor w.r.t. the inertial frame expressed in the body fram \\
	$\boldsymbol{\beta}_{\boldsymbol{\omega}}$ \> Gyro biases \\
	$\boldsymbol{\beta}_{\bf{a}}$ \> Accelerometer biases \\
	$\boldsymbol{\theta}_b$ \> Attitude as a vector of Euler angles \\
	$g$ \> Earth's gravity \\
	$\rho$ \> Air density \\
	$T_s$ \> GPS sample time \\
	$1/k_{GPS}$ \> GPS time constant \\
	$\phi$ \> Roll \\
	$\theta$ \> Pitch \\
	$\psi_m$ \> Magnetic North \\
	$\delta_d$ \> Magnetic declination \\
	$R_a^b$ \> Rotation from reference frame $a$ to $b$ \\
	$R_{\bf z}$ \> Measurement covariance \\
	$\boldsymbol{\eta}$ \> White noise \\
	$\hat{a}$ \> Estimate of true variable $a$ \\
	$\dot{a}$ \> Time derivative variable $a$ \\
	$\delta a$ \> Error state of variable $a$ \\
	[5pt]
	\textit{Superscript}\\
	$\top$ \> Matrix transpose \\
\end{tabbing}



\section{Conventions}

The following conventions are taken from \cite{key-2}. Quaternions take the form
\begin{equation}
{\bf{q}} = q_w + q_x{\bf{i}} + q_y{\bf{j}} + q_z{\bf{k}} = \begin{bmatrix} \bar{\bf{q}}^\top & q_w \end{bmatrix}^\top,
\end{equation}
where $\bar{\bf{q}} = \begin{bmatrix} q_x & q_y & q_z \end{bmatrix}^\top$.

Quaternion multiplication is defined by
\begin{equation}
{\bf{p}}\otimes{\bf{q}} = \begin{bmatrix} p_w\bf{I} + \left\lfloor \bar{\bf{p}} \right\rfloor & \bar{\bf{p}} \\ -\bar{\bf{p}}^\top & p_w \end{bmatrix} \begin{bmatrix} \bar{\bf{q}} \\ q_w \end{bmatrix},
\end{equation}
and the operator $\left\lfloor\cdot\right\rfloor$ is the skew-symmetric operator defined by
\begin{equation}
\left\lfloor\bf{a}\right\rfloor = \begin{bmatrix} 0 & -a_z & a_y \\ a_z & 0 & -a_x \\ -a_y & a_x & 0 \end{bmatrix}.
\end{equation}

Coordinate systems used in this paper are defined in \cite{key-3}. A $3\times3$ rotation matrix that rotates vectors from the vehicle coordinate system to the body coordinate system may be defined as
\begin{equation}
R\left(\bf{q}\right) = \left(2q_w^2-1\right){\bf{I}} - 2q_w\left\lfloor\bar{\bf{q}}\right\rfloor + 2\bar{\bf{q}}\bar{\bf{q}}^\top.
\end{equation}

This rotation matrix may also be defined as a function of Euler angles by
\begin{equation}
R\left(\bf{q}\right) = R_{v_2}^b\left(\phi\right)R_{v_1}^{v_2}\left(\theta\right)R_v^{v_1}\left(\psi\right),
\end{equation}
where
\begin{align}
R_{v_2}^b\left(\phi\right) & = \begin{bmatrix} 1 & 0 & 0 \\ 0 & \cos\phi & \sin\phi \\ 0 & -\sin\phi & \cos\phi \end{bmatrix} \\
R_{v_1}^{v_2}\left(\theta\right) & = \begin{bmatrix} \cos\theta & 0 & -\sin\theta \\ 0 & 1 & 0 \\ \sin\theta & 0 & \cos\theta \end{bmatrix} \\
R_v^{v_1}\left(\psi\right) & = \begin{bmatrix} \cos\psi & \sin\psi & 0 \\ -\sin\psi & \cos\psi & 0 \\ 0 & 0 & 1 \end{bmatrix},
\end{align}
and where the conversion from quaternions to Euler angles is given by
\begin{align}
\phi & = \mathrm{atan2}\left( \frac{2q_w q_x+2q_y q_z}{q_w^2+q_z^2-q_x^2-q_y^2} \right) \\
\theta & = \mathrm{asin}\left( 2q_w q_y - 2q_x q_z \right) \\
\psi & = \mathrm{atan2}\left( \frac{2q_w q_z+2q_x q_y}{q_w^2+q_x^2-q_y^2-q_z^2} \right).
\end{align}

Using the Gibbs vector parameterization, we can transform between Euler angle and quaternion attitude representations by
\begin{align}
\boldsymbol{\theta} & = {\bf q}^{\lor} = 2\bar{{\bf q}} \\
{\bf q} & = \boldsymbol{\theta}^{\land} = \frac{1}{\sqrt{4+\boldsymbol{\theta}^\top \boldsymbol{\theta}}} \begin{bmatrix} \boldsymbol{\theta} \\ 2 \end{bmatrix}.
\end{align}



\section{State Dynamics}

Derivations for this section are found in \cite{key-2}. The vehicle state is defined by
\begin{equation}
{\bf x}=\begin{bmatrix}{\bf p}_{b}^{\top} & {\bf q}_{b}^{\top} & {\bf v}_{b}^{\top} & \boldsymbol{\beta}_{\boldsymbol{\omega}}^{\top} & \boldsymbol{\beta}_{{\bf a}}^{\top}\end{bmatrix}^{\top},
\end{equation}
and its estimated dynamics w.r.t. a fixed frame are defined by
\begin{align}
\dot{\hat{{\bf p}}}_b & =R^{\top}\left(\hat{{\bf q}}\right)\hat{{\bf v}}\\
\dot{\hat{{\bf q}}}_b & =\dfrac{1}{2}\hat{{\bf q}}\otimes\begin{bmatrix}\hat{\boldsymbol{\omega}}\\0\end{bmatrix}\\
\dot{\hat{{\bf v}}}_b & =\left\lfloor \hat{{\bf v}}\right\rfloor \hat{\boldsymbol{\omega}}+R\left(\hat{{\bf q}}\right){\bf g}+{\bf k}{\bf k}^{\top}\hat{{\bf a}}\\
\dot{\hat{\boldsymbol{\beta}}}_{\boldsymbol{\omega}} & ={\bf 0}\\
\dot{\hat{\boldsymbol{\beta}}}_{{\bf a}} & ={\bf 0},
\end{align}
where
\begin{align}
{\bf k} & = \begin{bmatrix} 0 & 0 & 1 \end{bmatrix}^{\top}\\
{\bf g} & = \begin{bmatrix} 0 & 0 & g \end{bmatrix}^{\top}.
\end{align}



\section{Propagation}

Let's define a vector state and quaternion state as
\begin{align}
{\bf x}_{\bf v} & = \begin{bmatrix}{\bf p}_{b}^{\top} & {\bf v}_{b}^{\top} & \boldsymbol{\beta}_{\boldsymbol{\omega}}^{\top} & \boldsymbol{\beta}_{{\bf a}}^{\top}\end{bmatrix}^{\top} \\
{\bf x}_{\bf q} & = {\bf q}_{b}.
\end{align}

The estimated state dynamics are defined by
\begin{equation}
\dot{\hat{\bf x}} = f\left( \hat{\bf x}, {\bf u} \right) = \begin{bmatrix} \dot{\hat{\bf p}}_b^\top & \dot{\hat{\bf q}}_b^\top & \dot{\hat{\bf v}}_b^\top & \dot{\hat{\boldsymbol \beta}}_{\boldsymbol \omega}^\top & \dot{\hat{\boldsymbol \beta}}_{\bf a}^\top \end{bmatrix}^\top,
\end{equation}
which are used to propagate the estimated state forward in time via Euler integration. The vector state is propagated by simply adding small changes, but the unit quaternion part of the state must be propagated on forward on its manifold \cite{key-1}. Thus, forward propagation of the state is completed by
\begin{align}
\hat{{\bf x}}_{\bf v}\left(t+\Delta t\right) & = \hat{\bf x}_{\bf v}\left(t\right) + \delta_{\bf v} \\
\hat{{\bf x}}_{\bf q}\left(t+\Delta t\right) & = \hat{{\bf x}}_{\bf q}\left(t\right) \otimes \exp_q \frac{\delta_{\boldsymbol \theta}}{2},
\end{align}
where
\begin{align}
\delta_{\bf v} & = \dot{\hat{{\bf x}}}_{\bf v} \cdot \Delta t \\
\delta_{\boldsymbol \theta} & = \left(\dot{\hat{{\bf x}}}_{\bf q} \cdot \Delta t \right)^\lor \\
\exp_q{\delta} & = \begin{bmatrix} \frac{\sin{\lVert \delta \rVert}}{\lVert \delta \rVert}\delta \\ \cos{\lVert \delta \rVert} \end{bmatrix}.
\end{align}

The error state is given by
\begin{equation}
\delta{\bf x}=\begin{bmatrix}\delta{\bf p}_{b}^{\top} & \delta{\boldsymbol \theta}_{b}^{\top} & \delta{\bf v}_{b}^{\top} & \delta\boldsymbol{\beta}_{\boldsymbol{\omega}}^{\top} & \delta\boldsymbol{\beta}_{{\bf a}}^{\top}\end{bmatrix}^{\top},
\end{equation}
but is not propagated forward in time because its expected value is zero. However, its covariance must be propagated. The error state covariance derivative is given by
\begin{equation}
\dot{P} = FP + PF^\top + GQ_{\bf u} G^\top + Q_{\bf x},
\end{equation}
where the Jacobians are given by
\begin{align}
F & = 
\begin{bmatrix}
{\bf 0} & -R^\top \left(\hat{{\bf q}}_b \right) \lfloor \hat{{\bf v}}_b \rfloor & R^\top \left(\hat{{\bf q}}_b\right) & {\bf 0} & {\bf 0} \\
{\bf 0} & -\lfloor \hat{\boldsymbol{\omega}} \rfloor & {\bf 0} & -I & {\bf 0} \\
{\bf 0} & \lfloor R\left(\hat{{\bf q}}_b\right) {\bf g} \rfloor & -\lfloor \hat{\boldsymbol{\omega}} \rfloor & -\lfloor \hat{{\bf v}}_b \rfloor & -{\bf k}{\bf k}^\top \\
{\bf 0} & {\bf 0} & {\bf 0} & {\bf 0} & {\bf 0} \\
{\bf 0} & {\bf 0} & {\bf 0} & {\bf 0} & {\bf 0}
\end{bmatrix} \\
G & = 
\begin{bmatrix}
{\bf 0} & {\bf 0} \\
-I & {\bf 0} \\
-\lfloor \hat{{\bf v}}_b \rfloor & -{\bf k}{\bf k}^\top \\
{\bf 0} & {\bf 0} \\
{\bf 0} & {\bf 0}
\end{bmatrix}.
\end{align}
The matrix $Q_{\bf u}$ is the covariance of the gyro and accelerometer input ${\bf u} = \begin{bmatrix} \tilde{\boldsymbol{\omega}}^\top & \tilde{{\bf a}}^\top \end{bmatrix}^\top$, and $Q_{\bf x}$ is the process noise covariance. The error state covariance is also propagated with Euler integration by
\begin{equation}
P\left( t+\Delta t \right) = P\left(t\right) + \dot{P}\cdot \Delta t.
\end{equation}


\section{Update}

The measurement residual error for a general measurement with measurement model ${\bf h}$ is computed by
\begin{equation}
{\bf r} = {\bf z} - {\bf h}\left(\hat{{\bf x}}\right),
\end{equation}
where ${\bf z}$ is a sensor measurement.

The measurement uncertainty is computed by
\begin{equation}
S = HPH^\top + R_{\bf z},
\end{equation}
and the Kalman gain is then given by
\begin{equation}
K = PH^\top S^{-1}.
\end{equation}

The update is usually given by
\begin{equation}
\delta {\bf x}^+ = \delta {\bf x} + K{\bf r},
\end{equation}
but the quaternion portion cannot be correctly updated with addition. Separating $K{\bf r}$ into a vector state $\Delta{\bf v}$ and an attitude state $\Delta{\boldsymbol \theta}$, we perform the update by
\begin{align}
\hat{{\bf x}}_{\bf v}^+ & = \hat{{\bf x}}_{\bf v} + \Delta{\bf v} \\
\hat{{\bf x}}_{\bf q}^+ & = \hat{{\bf x}}_{\bf q}\exp_q{\frac{\Delta{\boldsymbol \theta}}{2}}.
\end{align}

The covariance is updated by
\begin{equation}
P^+ = \left( I - KH \right) P.
\end{equation}



\section{Sensor Models}

\subsection{IMU}

The gyro and accelerometer measurement models are given by
\begin{align}
\tilde{\boldsymbol{\omega}} & = \boldsymbol{\omega} + \boldsymbol{\beta}_{\boldsymbol{\omega}} + \boldsymbol{\eta}_{\boldsymbol{\omega}} \\
\tilde{\bf{a}} & = {\bf{a}} + \boldsymbol{\beta}_{\bf{a}} + \boldsymbol{\eta}_{\bf{a}},
\end{align}
and their best estimates are
\begin{align}
\hat{\boldsymbol{\omega}} & = \tilde{\boldsymbol{\omega}} - \hat{\boldsymbol{\beta}}_{\boldsymbol{\omega}} \\
\hat{\bf{a}} & = \tilde{\bf{a}} - \hat{\boldsymbol{\beta}}_{\bf{a}}.
\end{align}

These estimates are used as inputs to propagate the state estimate.

\subsection{Sonar}

The sonar measurement model is given by
\begin{equation}
h_{sonar} = -p_{b,d} + \eta_{sonar},
\end{equation}
and its error model is then given by
\begin{align}
r_{sonar} & = h_{sonar}\left({\bf x}\right) - h_{sonar}\left(\hat{{\bf x}}\right) \\
& = -p_{b,d} + \eta_{sonar} - \left(-\hat{p}_{b,d}\right) \\
& = -\delta p_{b,d} + \eta_{sonar}.
\end{align}

The measurement Jacobian is then
\begin{align}
H_{sonar} & = \frac{\partial r_{sonar}}{\partial \delta {\bf x}} \\
& = \begin{bmatrix} -{\bf k}^\top & {\bf 0} & {\bf 0} & {\bf 0} & {\bf 0} \end{bmatrix}.
\end{align}

\subsection{Barometer}

ROSflight contains an altitude lookup table for pressure measurements coming from the barometer, so its measurement model is the same as the sonar model, if we use the altitude estimate published from ROSflight.

\subsection{Magnetometer}

The magnetometer measures a magnetic field vector ${\bf m}_0$ in the body reference frame \cite{key-3}. Pitch and roll of the aircraft is removed from the measurement by
\begin{equation}
{\bf m} = R_{v_1}^{v_2}\left(\theta\right)^\top R_{v_2}^b\left(\phi\right)^\top {\bf m}_0,
\end{equation}
and the magnetic heading measurement is then given by
\begin{equation}
\psi_m = -\mathrm{atan2} \left( m_y, m_x \right).
\end{equation}

From eq. 7.11 in \cite{key-3}, heading is given by
\begin{equation}
\psi = \delta_d + \psi_m,
\end{equation}
where $\delta_d\approx0.1986$ in Provo, UT. 

The magnetometer measurement model is given by
\begin{equation}
h_{mag} = \psi + \beta_{mag} + \eta_{mag},
\end{equation}
and its error model is then given by
\begin{align}
r_{mag} & = h_{mag}\left({\bf x}\right) - h_{mag}\left(\hat{{\bf x}}\right) \\
& = \psi + \beta_{mag} + \eta_{mag} - \left( \hat{\psi} - \beta_{mag}\right) \\
& = \delta \psi + \eta_{mag}.
\end{align}

Note that with calibration, $\beta_{mag}\approx0$, so it does not need to be estimated.

The measurement Jacobian is then
\begin{align}
H_{mag} & = \frac{\partial r_{mag}}{\partial \delta {\bf x}} \\
& = \begin{bmatrix} {\bf 0} & {\bf k}^\top & {\bf 0} & {\bf 0} & {\bf 0} \end{bmatrix}.
\end{align}

\subsection{GPS}

The GPS measurement model is given by
\begin{equation}
{\bf{h}}_{gps}\left[n\right] = \begin{bmatrix} p_{b,n} + \eta_n \\ p_{b,e} + \eta_e \\ p_{b,d} + \eta_d \\ V_g + \eta_{V_g} \\ \psi + \eta_{\psi} \end{bmatrix},
\end{equation}
where 
\begin{align}
{\bf v}_g & =  I_{2\times3}R^\top \left({\bf q}_b\right){\bf v}_b \\
I_{2\times3} & = \begin{bmatrix} 1 & 0 & 0 \\ 0 & 1 & 0 \end{bmatrix} \\
V_g & = \sqrt{{\bf v}_g^\top {\bf v}_g} \\
\psi & = \arctan2\left({\bf v}_{g,y}, {\bf v}_{g,x}\right),
\end{align}
and its error model is then given by
\begin{equation}
{\bf r}_{gps} = \begin{bmatrix} \delta p_{b,n} + \eta_n \\ \delta p_{b,e} + \eta_e \\ \delta p_{b,d} + \eta_d \\ \delta V_g + \eta_{V_g} \\ \delta \psi + \eta_{\psi} \end{bmatrix}.
\end{equation}

The ground speed and course error derivatives are difficult to compute analytically in terms of the error state. These will simply be computed numberically. The measurement Jacobian is given by
\begin{align}
H_{gps} & = \frac{\partial {\bf r}_{gps}}{\partial \delta {\bf x}} \\
& = \begin{bmatrix} {\bf i}^\top & {\bf 0} & {\bf 0} & {\bf 0} & {\bf 0} \\
{\bf j}^\top & {\bf 0} & {\bf 0} & {\bf 0} & {\bf 0} \\
{\bf k}^\top & {\bf 0} & {\bf 0} & {\bf 0} & {\bf 0} \\
{\bf 0} & \frac{\partial {\bf r}_{gps,V_g}}{\partial \delta {\boldsymbol \theta}_b} & \frac{\partial {\bf r}_{gps,V_g}}{\partial \delta {\bf v}_b} & {\bf 0} & {\bf 0} \\
{\bf 0} & \frac{\partial {\bf r}_{gps,{\boldsymbol \theta}_b}}{\partial \delta {\boldsymbol \theta}_b} & \frac{\partial {\bf r}_{gps,{\boldsymbol \theta}_b}}{\partial \delta {\bf v}_b} & {\bf 0} & {\bf 0} \end{bmatrix},
\end{align}
where
\begin{align}
{\bf i} & = \begin{bmatrix} 1 & 0 & 0 \end{bmatrix}^\top \\
{\bf j} & = \begin{bmatrix} 0 & 1 & 0 \end{bmatrix}^\top.
\end{align}

\subsection{Attitude}

The flight controller only estimates attitude and tends to do it well, so we can use its attitude estimate to update roll and pitch. The attitude measurement model is given by
\begin{equation}
{\bf h}_{att} = \begin{bmatrix} \phi_b + \eta_{\phi_b} \\ \theta_b + \eta_{\theta_b} \end{bmatrix},
\end{equation}
and its error model is then given by
\begin{align}
{\bf r}_{att} & = {\bf h}_{att}\left({\bf x}\right) - {\bf h}_{att}\left(\hat{{\bf x}}\right) \\
& = \begin{bmatrix} \delta \phi_b + \eta_{\phi_b} \\ \delta \theta_b + \eta_{\theta_b} \end{bmatrix}.
\end{align}

The measurement Jacobian is then
\begin{align}
H_{att} & = \frac{\partial {\bf r}_{att}}{\partial \delta {\bf x}} \\
& = \begin{bmatrix} {\bf 0} & {\bf i}^\top & {\bf 0} & {\bf 0} & {\bf 0} \\
{\bf 0} & {\bf j}^\top & {\bf 0} & {\bf 0} & {\bf 0} \end{bmatrix}.
\end{align}



\begin{thebibliography}{1}

\bibitem{key-1}Hertzberg, Christoph, et al. \textquotedbl{}Integrating
generic sensor fusion algorithms with sound state representations
through encapsulation of manifolds.\textquotedbl{} Information Fusion
14.1 (2013): 57-77.

\bibitem{key-2}Wheeler and Koch. \textquotedbl{}Derivation of the Relative Multiplicative
Extended Kalman Filter\textquotedbl{}

\bibitem{key-3}Beard, Randal W., and Timothy W. McLain. \textquotedbl{}Small unmanned aircraft: Theory and practice\textquotedbl{}. Princeton university press, 2012.

\end{thebibliography}



\end{document}

